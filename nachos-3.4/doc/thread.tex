\documentstyle{article}

\begin{document}

\begin{center}
{\large Nachos Assignment \#1: Build a thread system

\vspace{.2in}
Tom Anderson\\
Computer Science 162\\
Due date: Tuesday, Sept. 21, 5:00 p.m.
}
\end{center}

\vspace{.2in}

In this assignment, we give you part of a working thread system;
your job is to complete it, and then to use it to solve
several synchronization problems.

The first step is to read and understand the partial thread system
we have written for you.  This thread system implements thread fork,
thread completion, along with semaphores for synchronization.
Run the program `nachos' for a simple test of our code.
Trace the execution path (by hand) for the simple test case
we provide.

When you trace the execution path, it is helpful to keep track
of the state of each thread and which procedures are on each thread's
execution stack.
You will notice that when one thread calls SWITCH, another thread
starts running, and the first thing the new thread does is
to return from SWITCH.
We realize this comment will seem cryptic to you at this point, but you
will understand threads once you understand
why the SWITCH that gets called is different from the SWITCH that returns.
(Note: because gdb does not understand threads, you will get bizarre
results if you try to trace in gdb across a call to SWITCH.)

The files for this assignment are:

\begin{description}

\item main.cc, threadtest.cc --- a simple test of our thread routines.

\item thread.h, thread.cc --- thread data structures and
thread operations such as thread fork, thread sleep and thread finish.

\item scheduler.h, scheduler.cc --- manages the list of threads that
are ready to run.

\item synch.h, synch.cc --- synchronization routines: semaphores, locks,
and condition variables.

\item list.h, list.cc --- generic list management (LISP in C++).

\item synchlist.h, synchlist.cc --- synchronized access to lists using
locks and condition variables (useful as an example of the use
of synchronization primitives).

\item system.h, system.cc --- Nachos startup/shutdown routines.

\item utility.h, utility.cc --- some useful definitions and debugging routines.

\item switch.h, switch.s --- assembly language magic for starting
up threads and context switching between them.

\item interrupt.h, interrupt.cc --- manage enabling and disabling
interrupts as part of the machine emulation.

\item timer.h, timer.cc --- emulate a clock that periodically causes
an interrupt to occur.

\item stats.h -- collect interesting statistics.

\end{description}

Properly synchronized code should work no matter what order the
scheduler chooses to run the threads on the ready list.  In other
words, we should be able to put a call to Thread::Yield (causing the scheduler
to choose another thread to run) anywhere in your code where interrupts
are enabled without changing the correctness of your code.
You will be asked to write properly synchronized code as part of the
later assignments, so understanding how to do this is crucial to
being able to do the project.

To aid you in this, code linked in with Nachos will cause Thread::Yield
to be called on your behalf in a repeatable but unpredictable way.
Nachos code is repeatable in that if you call it repeatedly with the
same arguments, it will do exactly the same thing each time.
However, if you invoke ``nachos -rs \#'', with a different number each
time, calls to Thread::Yield will be inserted at different places in the code.

Make sure to run various test cases against your solutions to
these problems; for instance, for part two, create multiple producers
and consumers and demonstrate that the output can vary, within certain
boundaries.

Warning: in our implementation of threads, each thread is assigned a
small, fixed-size execution stack.  This may cause bizarre problems
(such as segmentation faults at strange lines of code) if you declare
large data structures to be automatic variables (e.g., ``int buf[1000];'').
You will probably not notice this during the semester, but if you do,
you may change the size of the stack by modifying the StackSize define in
switch.h.

Although the solutions can be written as normal C routines, you will
find organizing your code to be easier if you structure your code
as C++ classes.  Also, there should be no busy-waiting in any of your
solutions to this assignment.

The assignment is items 1, 2, 5, 6 and 7 listed below.

\begin{description}

\item{1.}
Implement locks and condition variables.  You may either use semaphores
as a building block, or you may use more primitive thread routines (such
as Thread::Sleep).  We have provided the public interface to locks
and condition variables in synch.h.  You need to define the private
data and implement the interface.  Note that it should not take you very
much code to implement either locks or condition variables.

\item{2.}
Implement producer/consumer communication through a bounded buffer,
using locks and condition variables.  (A solution to the bounded buffer
problem is described in Silberschatz, Peterson and Galvin, using
semaphores.)

The producer places characters from the string "Hello world" into the
buffer one character at a time; it must wait if the buffer is full.
The consumer pulls characters out of the buffer one at a time
and prints them to the screen; it must wait if the buffer is empty.
Test your solution with a multi-character buffer and with multiple
producers and consumers.  Of course, with multiple producers or consumers,
the output display will be gobbledygook; the point is to illustrate

\item{3.} The local laundromat has just entered the computer age.
As each customer enters, he or she puts coins into
slots at one of two stations and types in
the number of washing machines he/she will need.  The stations
are connected to a central computer that
automatically assigns available machines and outputs
tokens that identify the machines to be used.  The
customer puts laundry into the machines and inserts each
token into the machine indicated on the token.  When a machine finishes
its cycle, it informs the computer that it is available again.
The computer maintains an array {\em available[NMACHINES]} whose elements are
non-zero if the corresponding machines are available (NMACHINES is
a constant indicating how many machines there are in the
laundromat), and a semaphore {\em nfree} that indicates how many
machines are available.

The code to allocate and release machines is as follows:

\begin{verbatim}
int allocate()	/* Returns index of available machine.*/
{
  int i;

  P(nfree);	/* Wait until a machine is available */
  for (i=0; i < NMACHINES; i++)
    if (available[i] != 0) {
      available[i] = 0;
      return i;
    }
}

release(int machine)	/* Releases machine */
{
  available[machine] = 1;
  V(nfree);
}
\end{verbatim}

The {\em available} array is initialized to all ones, and {\em nfree} is
initialized to NMACHINES.

\begin{description}

\item{(a)} It seems that
if two people make requests at the two stations at the same time, they
will occasionally be assigned the same machine.  This has resulted in
several brawls in the laundromat, and you have been called in by the
owner to fix the problem.  Assume that one thread handles each
customer station.  Explain how the same washing machine can be assigned
to two different customers.

\item{(b)} Modify the code to eliminate the problem.

\item{(c)} Re-write the code to solve the synchronization problem
using locks and condition variables instead of semaphores.

\end{description}

\item{4.} Implement an ``alarm clock'' class.
Threads call ``Alarm::GoToSleepFor(int howLong)'' to go to sleep for
a period of time.  The alarm clock can be implemented using the hardware
Timer device (cf. timer.h). When the timer interrupt goes off, the
Timer interrupt
handler checks to see if any thread that had been asleep needs to wake up now.
There is no requirement that threads start running immediately after
waking up; just put them on the ready queue after they have waited for
the approximately the right amount of time.

\item{5.} You've been hired by the University to build a controller
for the elevator in Evans Hall, using semaphores or condition variables.
The elevator is represented as a thread; each student or faculty member
is also represented by a thread.   In addition to the elevator manager,
you need to implement the routines called by the arriving student/faculty:
``ArrivingGoingFromTo(int atFloor, int toFloor)''.   This should wake
up the elevator, tell it which floor the person is on, and wait until
the elevator arrives before telling it which floor to go to.
The elevator is amazingly fast, but it is not instantaneous -- it takes
only 100 ticks to go from one floor to the next.   You may find it useful
to use your solution for part 4 here.  For simplicity, you can assume
there's only one elevator, and that it holds an arbitrary number of people.
Of course, you should give priority to those threads
going to or departing from the fifth floor :-).

\item{6.} You've just been hired by Mother Nature to help her out with the
chemical reaction to form water, which she doesn't seem to be
able to get right due to synchronization problems.  The trick is to
get two H atoms and one O atom all together at the same time.  The
atoms are threads.  Each H atom invokes a procedure {\em hReady} when it's
ready to react, and each O atom invokes a procedure {\em oReady}
when it's ready.  For this problem, you are to write the code for
{\em hReady} and {\em oReady}.  The procedures must delay until there
are at least two H atoms and one O atom present, and then one
of the procedures must call the procedure {\em makeWater} (which
just prints out a debug message that water was made).
After the {\em makeWater} call, two
instances of {\em hReady} and one instance of {\em oReady} should return.
Write the code for {\em hReady} and {\em oReady} using
either semaphores or locks and condition variables for synchronization.
Your solution must avoid starvation and busy-waiting.

\item{7.} You have been hired by Caltrans to synchronize traffic over a
narrow light-duty bridge on a public highway.  Traffic may only cross
the bridge in one direction at a time, and if there are ever more than
3 vehicles on the bridge at one time, it will collapse  under their
weight.  In this system, each car is represented by one thread,
which executes the procedure {\em OneVehicle} when it arrives
at the bridge:

\begin{verbatim}
OneVehicle(int direc)
{
  ArriveBridge(direc);
  CrossBridge(direc);
  ExitBridge(direc);
}
\end{verbatim}

In the code above, {\em direc} is either 0 or 1;  it gives the
direction in which the vehicle will cross the bridge.

\begin{description}

\item{(a)} Write the procedures {\em ArriveBridge}
and {\em ExitBridge} (the {\em CrossBridge} procedure should just print
out a debug message), using locks and condition variables.  {\em ArriveBridge}
must not return until it safe
for the car to cross the bridge in the given direction (it
must guarantee that there will be no head-on collisions or
bridge collapses).  {\em ExitBridge} is called to indicate
that the caller has finished crossing the bridge;  {\em ExitBridge}
should take steps to let additional cars cross the bridge.
This is a lightly-travelled rural bridge, so
you do not need to guarantee fairness or freedom from starvation.

\item{(b)} In your solution, if a car arrives while traffic is currently moving
in its direction of travel across the bridge, but there is another
car already waiting to cross in the opposite direction, will the
new arrival cross {\em before} the car waiting on the other side,
{\em after} the car on the other side, or is it impossible to
say?  Explain briefly.

\end{description}

\item{8.}
Implement (non-preemptive) priority scheduling.
Modify the thread scheduler to always return the
highest priority thread.  (You will need to create a new constructor
for Thread to take another parameter -- the priority level of the thread;
leave the old constructor alone since we'll need it for backward
compatibility.)  You may assume that there are a fixed, small
number of priority levels -- for this assignment, you'll only need two levels.

Can changing the relative priorities of the producers and consumer
threads have any affect on the output?  For instance, what happens with
two producers and one consumer, when one of the
producers is higher priority than the other?  What if the two
producers are at the same priority, but the consumer is at high
priority?

\item{9.} You have been hired to simulate one of the Cal fraternities.
Your job is to write a computer program to pair up men and women
as they enter a Friday night mixer.  Each man and each woman will
be represented by one thread.  When the man or woman enters the
mixer, its thread will call one of two procedures, {\em man}
or {\em woman}, depending on the sex of the thread.  You must
write C code to implement these procedures.  Each procedure takes
a single parameter, {\em name}, which is just an integer name for
the thread.  The procedure must wait until there is an available
thread of the opposite sex, and must then exchange names
with this thread.
Each procedure must return the integer name of the thread
it paired up with.  Men and women may enter the fraternity
in any order, and many threads may call the {\em man} and
{\em woman} procedures simultaneously.  It doesn't matter
which man is paired up with which woman (Cal frats aren't
very choosy), as long as each pair contains one man and one woman
and each gets the other's name.  Use semaphores and shared
variables to implement the two procedures.  Be sure to
give initial values for the semaphores and indicate which
variables are shared between the threads.  There must
not be any busy-waiting in your solution.

\item{10.} Implement the synchronization for
a ``lockup-free'' cache, using condition variables.
A lockup-free cache is one that can continue to accept requests
even while it is waiting for a response from memory (or equivalently, the
disk).  This is useful, for instance, if the processor can pre-fetch data
into the cache before it is needed; this hides memory latency
only if it does not interfere with normal cache accesses.

The behavior of a lockup-free cache can be modelled with threads,
where each thread can ask the cache to read or write the data at
some physical memory location.  For a read, if the data is cached,
the data can be immediately returned.  If the data is not cached,
the cache must (i) kick something out of the cache to clear space
(potentially having to write it back to physical memory if it
is dirty), (ii) ask memory to fetch the item, and (iii) when the
data returns, put it in the cache and return the data to the
original caller.  The cache stores data in one unit chunks, so
a write request need not read the location in before over-writing.
While memory is being queried, the cache can accept requests from
other threads.   Of course, the cache is fixed size, so it is
possible (although unlikely) that all items in the cache may have
been kicked out by earlier requests.

You are to implement the routines {\em CacheRead(addr)} and
{\em CacheWrite(addr, val)}; these routines call {\em DiskRead(addr)}
and {\em DiskWrite(addr, val) on a cache miss} -- you can assume
these disk operations are already implemented.

\item{11.} You have been hired by the CS Division to write code to help
synchronize a professor and his/her students during office hours.
The professor, of course, wants to take a nap if no students are around to
ask questions; if there are students who want to ask questions,
they must synchronize with each other and with the professor so that
(i) only one person is speaking at any one time, (ii) each
student question is answered by the professor, and (iii) no student
asks another question before the professor is done answering the previous
one.  You are to write four procedures: {\em AnswerStart(), AnswerDone(),
QuestionStart(), and QuestionDone()}.
The professor loops running the code: AnswerStart(); give answer; AnswerDone().
AnswerStart doesn't return until a question has been asked.
Each student loops running the code: QuestionStart(); ask question;
QuestionDone().  QuestionStart() does not return until it is the student's
turn to ask a question.   Since professors consider it rude for a student
not to wait for an answer, QuestionEnd() should not return until the
professor has finished answering the question.

\item{12.} You have been hired by Greenpeace to help the environment.
Because unscrupulous commercial interests have dangerously lowered
the whale population, whales are having synchronization problems
in finding a mate.  The trick is that in order to have children,
{\em three} whales are needed, one male, one female, and one to
play matchmaker -- literally, to push the other two whales together
(I'm not making this up!).
Your job is to write the three procedures {\em Male(), Female(), and
Matchmaker()}.  Each whale is represented by a separate thread.
A male whale calls Male(), which waits until there is a waiting female
and matchmaker; similarly, a female whale must wait until a male whale
and a matchmaker are present.  Once all three are present, all three return.

\item{13.}
A particular river crossing is shared by both
cannibals and missionaries.  A boat is used to cross the river, but
it only seats three people, and must always carry a full load.  In
order to guarantee the safety of the missionaries, you cannot put
one missionary and two cannibals in the same boat (because the cannibals would
gang up and eat the missionary), but all other combinations are legal.
You are to write two procedures: {\em MissionaryArrives} and
{\em CannibalArrives},
called by a missionary or cannibal when it arrives at the river bank.
The procedures arrange the arriving missionaries and cannibals into
safe boatloads; once the boat is full, one thread calls {\em RowBoat}
and after the call to {\em RowBoat}, the three procedures then return.
There should also be no undue waiting:
missionaries and cannibals should not wait if there are enough of them
for a safe boatload.

\item{14.}
You have been hired by your local bank to solve the {\em safety deposit box}
synchronization problem.  In order to open a safety deposit box,
you need {\em two} keys to be inserted simultaneously, one from
the customer and one from the bank manager.  (For those of you who
saw Terminator 2, something like this protected access to the
vault containing the Terminator's CPU chip.)  The customer and the
bank manager are threads.   You are to write two procedures:
{\em CustomerArrives()} and {\em BankManagerArrives()}.
The procedures must delay until both are present; the customer
then calls {\em OpenDepositBox()}.   In addition, the bank manager
must wait around until the customer finishes, to lock up the bank vault --
in other words, the bank manager cannot return from {\em BankManagerArrives()}
until after the customer has returned from {\em OpenDepositBox()}.
The bank manager can then take a coffee break, while the customer goes off to
spend the contents of the safety deposit box.

\end{description}

\end{document}
